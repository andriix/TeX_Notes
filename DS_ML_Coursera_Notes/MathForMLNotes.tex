\documentclass[16pt]{article}
%Packages
\usepackage[T1,T2A]{fontenc}
\usepackage[utf8]{inputenc}
\usepackage[english,ukrainian]{babel}
\usepackage{hyperref}
\usepackage{amsmath}
%Header information
\title{"Math for ML and DS" Specialization }
\author{ Andrii X }
\date{}

\begin{document}
	\maketitle
	
	\section{Linear Algebra for Machine Learning and Data Science}
	\subsection{System of linear equations:}
	\begin{itemize}
	\item {Systems of equations: \textbf{Non-singular} (complete), \textbf{Singular} (Redundant, Contradictory).}
	\item {\textbf{Determinant} of a matrix is the signed \textbf{factor by which areas are scaled by this matrix}. 
	\\
	For A = 
	$\begin{bmatrix}
		a & b\\
		c & d
	\end{bmatrix}$ 
	determinant is $\det(A) = ad-cb$. 
	\\
	For \textbf{non-singular} system determinant has \textbf{non-zero} value.
	\\
	Determinant of an \textbf{inverse matrix} is an \textbf{inverse of determinant} for original matrix: 
	$det(A^{-1})=\frac{1}{det(A)}$. 
	}
	\end{itemize}

	\subsection{Solving system of linear equations:}
	\begin{itemize}
	\item {\textbf{Rank} of a matrix tells how much information matrix has. For example, a matrix with 3 rows max rank is 3, since 3 eq and 3 pieces of information, but if one of eq is just a combination of 2 others then rank will be 2.
	\\
	Rank of a matrix can be calculated via \textbf{row echelon form}: 
	\begin{itemize}
		\item Zero rows at the bottom. 
		\item Each row has pivot (leftmost non-zero entry). 
		\item Every pivot is to the right of the pivots on the rows above. 
		\item Rank of the matrix is the number of pivots
	\end{itemize}
	Difference of Reduced REF from REF is that any number above a pivot is \textbf{0} in RREF. 
	}	
	\end{itemize}
	\subsection{Vectors and Linear Transformations:}
	\begin{itemize}
	\item {\textbf{Norm} is a function from vector space to the non-nega tive real numbers that behaves \textbf{like the distance} from the origin.}
	\end{itemize}
	
	
\end{document}