\documentclass[16pt]{article}
%Packages
\usepackage[T1,T2A]{fontenc}
\usepackage[utf8]{inputenc}
\usepackage[english,ukrainian]{babel}
\usepackage{hyperref}
%Header information
\title{ Dynamical systems, $\epsilon$-machines and complexity }
\author{ Andrii Khrinenko }
\date{}

%Main part
\begin{document}
	\maketitle	

\section{Вступ}

Показник складності Колмогорова є малопридатним для реалізації і практичного застосування. Для оцінки складності можна використовувати епсілон-машини.

* Цікава дисертація про епсілон-машини, використання логістичного тенту та практичне застосування обчислювальної механіки: \href{https://amslaurea.unibo.it/15649/1/mattia_barbaresi_tesi.pdf}{COMPUTATIONAL MECHANICS: from theory to practice} 

\section{Finite State Machines}

\( M=<Q,\Sigma,\delta,q_0,F> \), where: Q - states (\(q_0,...q_n\)), $\Sigma$ - alphabet, \(\delta:Q*\Sigma->Q\), $q_0$-initial state, \(F\subseteq Q\) - final states.

\subsection{Epsilon-machine}

An $\epsilon$-machine is a computational model of a natural process. An algorithm called $\epsilon$-machine reconstruction generates a machine from a given sequence of measurements of the process. The $\epsilon$-machine generated by the reconstruction algorithm is provably the unique, minimal machine at the least-powerful computational level that is an optimal predictor of the data \cite{shalizi2001computational}.

\bibliographystyle{plain}
\bibliography{refs}
\end{document}